
\subsection{Theoretical Background}
%TODO do and intro to section here

\subsubsection{The Mößbauer Effect}
When an atom at rest emits a photon, part of the released energy is used to satisfy momentum conservation and recoil the atom.
Similarly when absorbing a photon, it has to carry the energy for the excitation and the recoil.
Resonance absorption, where the absorber is of the same type as the emitter, can only occur, when the linewidth of the decay is larger than the recoil energy.\\

For visible light resonance absorption is possible, while it is impossible for nuclear $\gamma$-transitions.\\

In reality at finite Temperature of the samples the doppler effect increases the widths of the lines creating an overlap in emission and absorption spectra and enabling resonance absorption.\\

The Mößbauer effect occurs, when certain atoms (e.g. \ce{Fe^57})are built into a crystal structure. Atoms in the lattice can emit photons creating or consuming phonons, but there is a finite propability, that the internal energy of the crystal does not change.


\subsubsection{Hyperfine Structure}

\subsubsection{Quadrupole Splitting}

\subsubsection{Isomery Shift}

\subsection{Assignments}

\subsubsection{Mößbauer Spectra of different Iron Compounds}

\subsubsection{Lifetime of $14.4 \ \text{keV}$ State in Vacromium}

\subsubsection{Inner Magnetic Field and Magnetic Moment of the Excited State in Iron}

\subsubsection{Electric Field Gradient in Iron Compounds}

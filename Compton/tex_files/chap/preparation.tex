\subsection{Theoretischer Hintergrund}
\subsubsection{$\gamma$-Strahlung und Zerfall}
Es gibt verschiedene Arten von Strahlung. Im Praktikum treten davon die $\gamma$-Strahlung, $\alpha$-Strahlung, $\beta^{+}$- und $\beta^{-}$-Strahlung, sowie Neutronenstrahlung auf. Für den Compton-Effekt ist die $\gamma$-Strahlung oder EM-Strahlung wichtig. \\
Sie entsteht beim Zerfall von Atomen, wenn die Tochterkerne danach in einem angeregten Zustand sind. Bei dem Übergang in den Grundzustand wird Energie in Form von Photonen abgestrahlt. Das ist die $\gamma$-Strahlung.\\

\paragraph{innere Konversion}

\subsubsection{Wechselwirkung von Photonen mit Materie}

\subsubsection{Detektor}

\subsubsection{Versuchsaufbau}

\subsubsection{Aufgaben}

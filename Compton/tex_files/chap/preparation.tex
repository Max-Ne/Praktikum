\subsection{Theoretischer Hintergrund}
\subsubsection{$\gamma$-Strahlung und Zerfall}
Es gibt verschiedene Arten von Strahlung. Im Praktikum treten davon die $\gamma$-Strahlung, $\alpha$-Strahlung, $\beta^{+}$- und $\beta^{-}$-Strahlung, sowie Neutronenstrahlung auf. Für den Compton-Effekt ist die $\gamma$-Strahlung wichtig. \\
Sie entsteht beim Zerfall von Atomen, wenn die Tochterkerne danach in einem angeregten Zustand sind. Bei dem Übergang in den Grundzustand wird Energie in Form von Photonen abgestrahlt. Das ist die $\gamma$-Strahlung. \\
Sie kann nach Kugelfächenfunktionen, den Multipolen, entwickelt werden. Der L-te Multipol hat die Parität $(-1)^{L}$ und durch den Zerfall bedingt haben die elektrischen und magnetischen Multipole $(-1)^{L}$ bzw. $(-1)^{L+1}$. Somit ist die Ordnung des magnetischen Anteils um eine Ordnung geringer als der zugehörige elektrische Anteil und wird nur dann wesentlich beobachtet, wenn letzterer durch Auswahlregeln verboten ist. \\
Für den Drehimpuls gilt mit den Anfangs- und Endzuständen $J_i$ und $J_f$ die übliche Addition der Quantenmechanik:
$$\vert J_f - J_i \vert \leq L \leq \vert J_f + J_i \vert $$


\paragraph{innere Konversion} ???

\subsubsection{Wechselwirkung von Photonen mit Materie}

\subsubsection{Detektor}

\subsubsection{Versuchsaufbau}

\subsubsection{Aufgaben}

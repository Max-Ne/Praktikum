\subsection{Theoretical Background}
\subsubsection{$\gamma$-Radiation and Decay}
There are different kinds of radiation. In the experiments there are EM-radiation, $\alpha$-radiation, $\beta^{+}$- and $\beta^{-}$-radiation as well as radiation of neutrons. Important for compton scattering is the  $\gamma$-radiation, a part of the electro magnetic spectrum.\\
It is emitted when atoms decay and the new atom is in an excited state. Photons are emitted as the atom relaxes to the ground state.\\
The $\gamma$-radiation can be expanded in multipoles, i.e. a sum of spherical harmonics. The parity of the L-th multipole is$(-1)^{L}$. Induced by the decay the electric and magnetic part has the parity $(-1)^{L}$ resp. $(-1)^{L+1}$. Hence, the order of the magnetic part is higher by 1 than the electric, so that mainly the electric part is observed. \\
The standard rule to add two angular momentums in QM gives for the initial state $J_i$ and the final state $J_f$ of the relaxing atom:
$$\vert J_f - J_i \vert \leq L \leq \vert J_f + J_i \vert $$ 

\subsubsection{Interaction of Photons with Matter}
There are different effects that play a role when photons interact with matter. Which effect is dominant depends on the energy of the photon as well as the atomic number Z. All effects have in common that the photon is either absorbed or scattered, so that there are less photons after the interaction with matter in the initial direction. The intensity follows an exponential law:
$$I(d) = I(0) \cdot e^{-\mu \cdot d} $$
Compton scattering dominantes starting at about ~100 keV up to 10 MeV. \\
Photons with the energy $E$ are scattered inelasticly with a charged particle with the energy $E_0$. The photon loses some energy and changes its direction, the electron's kinetic energy increases. Conservation of 4-momentum gives the energy of the scattered photon as a function of the initial energy $E$ and the polar angle $\Theta$:
$$E' = \frac{E}{1 + \frac{E}{E_0} \cdot (1 - \cos \Theta)}$$

\newpage

\subsubsection{Cross Section}
The cross section $\sigma$ gives the probability of a particle to interact with a target. Referring to some statics it is defined as:
$$\sigma = \frac{N_{int} \cdot F}{N_{all} \cdot N_{targets}} $$
with the number of interacting particles $N_{int}$, the cross section the beam $F$, the total number of particles in the beam $N_{all}$ and the number of targets $N_{targets}$. It has the same unit as a plane and is often visualized as such. In fact the cross section of two colliding billiard balls is just there geometric cross section. In other cases the cross section decreases with the distance of the two objects, e.g. Rutherford Scattering which depends on the electromagnetic interaction.\\
The differential cross section with respect to the solid angle $\Omega$ is often used and it is assumed to be independent of the azimuthal angle $\Phi$.\\
There is a theoretical formula by Klein and Nishina for the cross section of compton scattering.

\subsubsection{Experimental Setup}
Monoenergetic $\gamma$-radiation is emitted by a $^{137}$Cs source. The target is a Al-Cylinder. The detector is made of NaJ and can be moved on a circle around the target. \\
We will measure the energy and the number of the photons after the scattering as a function of the polar angle $\Theta$. This is because we assume rotational symmetry.

\subsubsection{Assignment 1}
We meassure the differential cross section with respect to the solid angle. For this the detector measures the number of incoming photons. For evaluating we use following formula:
$$\frac{d\sigma}{d\Omega} = \frac{R(\Delta \Omega)}{\Delta \Omega} \cdot \frac{1}{\Phi _0 n} \cdot \frac{1}{\epsilon} $$
In this formula $\frac{R(\Delta \Omega)}{\Delta \Omega}$ is a approximation of detector surface because it is not a small dot. $R(\Delta \Omega)$ is the photon stream, the number of detected photons per time interval. We also devide by $\Phi _0$ the incoming photon stream. $\frac{1}{\epsilon}$ is a given number to desribe the not detected electrons and finally $n$ is the number of electrons in the target, which follows from the cylindric geometry, the desity and the atomic number:
$$n =  \frac{L}{A} \cdot Z \rho \pi (\frac{d}{2})^{2} \cdot l $$

\newpage

\subsubsection{Assignment 2}
In assignment 2 we meassure the energy of the photons after the scattering as a function of the angle $\Theta$ and the initial energy. \\
First, we must calibrate the system with known sources. Then we make the meassurement and use the formula above in a slightly different way. A linear regression gives us the invariant mass of electrons.
$$\frac{1}{E'} = \frac{1}{E} + \frac{1}{m_0 \cdot c^{2}} \cdot (1 - \cos \Theta) $$

\subsubsection{Assignment 3}
The above formula gives the differential cross section for a single electron. For the whole atom it should be proportional to the atomic number, given that the photon energy is large with respect to the binding energy. With this assumption and the above equations from assignment we can write
$$(\frac{\mathrm{d}\sigma}{\mathrm{d}\Omega})_e = const. \cdot R \frac{A}{\rho Z} $$
